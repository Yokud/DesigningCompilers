\documentclass[a4paper]{bmstu}

\begin{document}

\makereporttitle
{Информатика и системы управления (ИУ)}
{Программное обеспечение ЭВМ и информационные технологии (ИУ7)}
{лабораторной работе № 1}
{Конструирование компиляторов}
{Распознавание цепочек регулярного языка}
{6}
{И.~А.~Малышев/ИУ7-22М}
{А.~А.~Ступников}


\setcounter{page}{2}
% \renewcommand{\contentsname}{Содержание} 
% \tableofcontents


\chapter{Выполнение лабораторной работы}

\section{Задание}

Напишите программу, которая в качестве входа принимает произвольное регулярное выражение, и выполняет
следующие преобразования:
\begin{enumerate}
    \item Преобразует регулярное выражение непосредственно в ДКА.
    \item По ДКА строит эквивалентный ему КА, имеющий наименьшее возможное количество состояний.
    \item Моделирует минимальный КА для входной цепочки из терминалов исходной грамматики (воспользоваться алгоритмом минимизации ДКА Хопкрофта).
\end{enumerate}

\section{Набор тестов}

\begin{table}[H]
    \centering
	\caption{Набор тестов и ожидаемые результаты работы программы}
    \label{tbl:tests}
	\begin{tabular}{|c|c|c|c|}
        \hline
        \begin{minipage}[t]{4cm}\centering \textbf{Регулярное выражение}\end{minipage} &
        \begin{minipage}[t]{4cm}\centering \textbf{Входная цепочка}\end{minipage} &
        \begin{minipage}[t]{4cm}\centering \textbf{Ожидаемый результат}\end{minipage} &
        \begin{minipage}[t]{4cm}\centering \textbf{Результат}\end{minipage} \\ \hline
        a* & a & соответствует & соответствует \\ \hline
        a* & aaa & соответствует & соответствует \\ \hline
        a* & b & не соответствует & не соответствует \\ \hline
        a* & пустая & соответствует & соответствует \\ \hline

        (a|b)*abb & abb & соответствует & соответствует \\ \hline
        (a|b)*abb & aaabb & соответствует & соответствует \\ \hline
        (a|b)*abb & babaabb & соответствует & соответствует \\ \hline
        (a|b)*abb & ababbb & не соответствует & не соответствует \\ \hline
        (a|b)*abb & пустая & не соответствует & не соответствует \\ \hline
        
        ((aa)|(bb)|c)* & aabb & соответствует & соответствует \\ \hline
        ((aa)|(bb)|c)* & bbcccbbc & соответствует & соответствует \\ \hline
        ((aa)|(bb)|c)* & aacab & не соответствует & не соответствует \\ \hline
        ((aa)|(bb)|c)* & пустая & соответствует & соответствует \\ \hline
    \end{tabular}
\end{table}

\clearpage
\section{Результаты работы программы}

Результаты работы программы для регулярного выражения \texttt{(a|b)*abb} приведены на рисунках \ref{img:parse-tree.gv}--\ref{img:min-dfa.gv}.

\includeimage{parse-tree.gv}{f}{H}{0.5\textwidth}{Синтаксическое дерево для регулярного выражения}

\includeimage{firstpos-lastpos.gv}{f}{H}{0.8\textwidth}{Значения функций firstpos и lastpos в узлах синтаксического дерева}

\includeimage{followpos.gv}{f}{H}{0.245\textwidth}{Ориентированный граф для функции followpos}

\includeimage{dfa.gv}{f}{H}{0.285\textwidth}{ДКА для регулярного выражения}

\includeimage{min-dfa.gv}{f}{H}{0.45\textwidth}{Минимизированный ДКА алгоритмом Хопкрофта}


\chapter{Контрольные вопросы}

\begin{enumerate}
    \item Какие из следующих множеств регулярны? Для тех, которые регулярны, напишите регулярные выражения.
    \begin{enumerate}
        \item Множество цепочек с равным числом нулей и единиц.
        
        \textbf{Ответ:} Не является регулярным множеством.

        \item Множество цепочек из \texttt{\{0, 1\}*} с четным числом нулей и нечетным числом единиц.
        
        \textbf{Ответ:} Является регулярным множеством. 

        \textbf{Пример:} ((0110)|(1001)|(1010)|(0101)|(11)|(00))*1
        
        ((0110)|(1001)|(1010)|(0101)|(11)|(00))*

        \item Множество цепочек из \texttt{\{0, 1\}*}, длины которых делятся на 3.
        
        \textbf{Ответ:} Является регулярным множеством. 

        \textbf{Пример:} ((0|1)(0|1)(0|1))*

        \item Множество цепочек из \texttt{\{0, 1\}*}, не содержащих подцепочки 101.
        
        \textbf{Ответ:} Является регулярным множеством. 

        \textbf{Пример:} ((0*00)|1)*
    \end{enumerate}
    \item Найдите праволинейные грамматики для тех множеств из вопроса 1, которые регулярны.
    \begin{enumerate}
        \item \begin{equation}
            \begin{split}
                S \to 0110S \\
                S \to 1001S \\
                S \to 1010S \\
                S \to 0101S \\
                S \to 11S \\
                S \to 00S \\
                S \to 1A \\
                A \to 0110A \\
                A \to 1001A \\
                A \to 1010A \\
                A \to 0101A \\
                A \to 11A \\
                A \to 00A \\
                A \to \epsilon
            \end{split}
        \end{equation}
        \item \begin{equation}
            \begin{split}
                S \to 0A \\
                S \to 1A \\
                S \to \epsilon \\
                A \to 0B \\
                A \to 1B \\
                B \to 0S \\
                B \to 1S
            \end{split}
        \end{equation}
        \item \begin{equation}
            \begin{split}
                S \to A \\
                S \to 1S \\
                S \to \epsilon \\
                A \to 0A \\
                A \to 00S
            \end{split}
        \end{equation}
    \end{enumerate}
    \item Найдите детерминированные и недетерминированные конечные автоматы для тех множеств из вопроса 1,
    которые регулярны.
    \begin{enumerate}
        \item \includeimage{3a}{f}{H}{0.7\textwidth}{ДКА для первого регулярного выражения}
        \item \includeimage{3b}{f}{H}{0.3\textwidth}{ДКА для второго регулярного выражения}
        \item \includeimage{3c}{f}{H}{0.3\textwidth}{ДКА для третьего регулярного выражения}
    \end{enumerate}
    \item Найдите конечный автомат с минимальным числом состояний для языка, определяемого автоматом M = (\{A,
    B, C, D, E\}, \{0, 1\}, d, A, \{E, F\}), где функция в задается таблицей
    \includeimage{question4}{f}{H}{0.35\textwidth}{Таблица для 4 вопроса}
    \includeimage{4-dfa}{f}{H}{0.35\textwidth}{ДКА для языка, определяемого автоматом M}
    \includeimage{4-min-dfa}{f}{H}{0.35\textwidth}{Минимизированный ДКА для языка, определяемого автоматом M}
\end{enumerate}

\chapter{Текст программы}

В листингах 3.1--3.6 представлен код программы.

\includelisting{main.py}{Основной модуль программы}

\includelisting{regularExpression.py}{Модуль обработки регулярных выражений}

\includelisting{parseTree.py}{Модуль для работы с синтаксическим деревом регулярного выражения}

\includelisting{dfa.py}{Модуль для работы с ДКА}

\includelisting{minDfa.py}{Модуль для работы с минимизированным ДКА}

\includelisting{chain.py}{Модуль обработки входных цепочек}

\end{document}
